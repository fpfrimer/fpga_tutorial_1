\documentclass[12pt,a4paper]{article}

% Suporte para a língua portuguesa
\usepackage[utf8]{inputenc}
\usepackage[T1]{fontenc}
\usepackage[brazil]{babel}

% Melhorias de tipografia
\usepackage{microtype}

% Cores personalizadas
\usepackage{xcolor}
\definecolor{darkblue}{RGB}{25,25,112}

% Para cabeçalhos bonitos
\usepackage{fancyhdr}

% Controlar layout da página
\usepackage{geometry}
\geometry{
    a4paper,
    total={170mm,257mm},
    left=20mm,
    top=20mm,
}

% Para inserção e manipulação de imagens
\usepackage{graphicx}
\graphicspath{{imagens/}} % Define o diretório padrão das imagens
\usepackage{float} % Para melhor controle de posicionamento

% Para títulos de seções coloridos e personalizados
\usepackage{titlesec}
\titleformat{\section}
{\color{darkblue}\normalfont\Large\bfseries}
{\color{darkblue}\thesection}{1em}{}

\titleformat{\subsection}
{\color{darkblue}\normalfont\large\bfseries}
{\color{darkblue}\thesubsection}{1em}{}

% Para links e referências
\usepackage{hyperref}
\hypersetup{
    colorlinks=true,
    linkcolor=darkblue,
    filecolor=magenta,      
    urlcolor=cyan,
}
% Permite quebras de linha em URLs em qualquer lugar, melhorando a quebra de linhas longas
\usepackage{xurl}

\pagestyle{fancy}
\fancyhf{}
\rhead{\textbf{Lógica Reconfigurável/Tutorial 1}}
\lhead{\textbf{\leftmark}}
\cfoot{\thepage}

\title{\textbf{Criando e Simulando Circuitos Digitais no Quartus}}
\author{Prof. Felipe W. D. Pfrimer}
\date{\today}

\begin{document}

\maketitle
\tableofcontents
\newpage

\section{Introdução}

O IntelFPGA Quartus Prime representa uma plataforma abrangente de desenvolvimento para Dispositivos Lógicos Programáveis (PLDs), oferecida pela IntelFPGA. Este Ambiente de Desenvolvimento Integrado (IDE) capacita os desenvolvedores na criação, análise, e síntese de sistemas digitais, empregando Linguagens de Descrição de Hardware (HDLs) ou diagramas esquemáticos. A versão Lite do Quartus Prime, especificamente a 18.1, está disponível gratuitamente e pode ser obtida diretamente através do site da Intel, usando a macro LaTeX para URLs como segue: \url{https://www.intel.com.br/content/www/br/pt/products/details/fpga/development-tools/quartus-prime/resource.html}.

Este tutorial é direcionado a fornecer ao leitor uma introdução prática ao desenvolvimento e simulação de sistemas digitais utilizando diagramas esquemáticos no Quartus Prime Lite 18.1. Embora nosso foco esteja na versão 18.1, é importante notar que os princípios e procedimentos aqui descritos são amplamente aplicáveis a outras versões do software, proporcionando uma base sólida que transcende variações específicas de versão.

\section{Instalação do Quartus Prime Lite 18.1}

Este guia passo a passo destina-se a auxiliar na instalação do Quartus Prime Lite 18.1 em sistemas operacionais Windows 10 ou 11.

\subsection{Download do Software}

\begin{enumerate}
    \item Acesse a página oficial de downloads da Intel para o Quartus Prime Lite 18.1 através do link: \url{https://www.intel.com/content/www/us/en/software-kit/795188/intel-quartus-prime-lite-edition-design-software-version-18-1-for-windows.html}.
    \item Na página, procure pela seção de downloads do Quartus Prime Lite Edition.
    \item Selecione a versão apropriada (18.1) para Windows e clique no botão de download, na aba de \textit{Multiple Download}.
    \item Pode ser necessário criar uma conta Intel ou fazer login com uma conta existente para prosseguir com o download.
\end{enumerate}

\subsection{Instalação do Software}

\begin{enumerate}
    \item Uma vez concluído o download, localize o arquivo baixado e execute-o com um duplo clique.
    \item Se for exibida uma janela pedindo permissão para que o aplicativo faça alterações no seu dispositivo, clique em "Sim".
    \item Siga as instruções apresentadas pelo assistente de instalação. Aceite os termos de licença e selecione o diretório de instalação conforme desejado. De preferência, deixe o diretório proposto.
    \item Durante a instalação, você pode selecionar os componentes específicos do Quartus que deseja instalar (dispositivos). Para uma instalação padrão, é recomendável deixar as opções pré-selecionadas. Tenha certeza de manter a família MAX10 e o modelsim IntelFPGA Edition ou Altera Edition. O modelsim padrão é um aplicativo que exige a compra de uma licença.
    \item Após configurar suas preferências, prossiga com a instalação e aguarde até que o processo seja concluído. Isso pode demorar bastante, recomenda-se café \includegraphics[height=1em]{latte-art.png}.
\end{enumerate}

\subsection{Inicialização do Programa}

\begin{enumerate}
    \item Após a conclusão da instalação, você pode iniciar o Quartus Prime Lite 18.1 através do menu Iniciar do Windows, procurando por "Quartus Prime Lite" ou através do atalho criado na área de trabalho, se disponível.
    \item Na primeira execução, pode ser necessário configurar algumas preferências iniciais ou realizar o registro do software, dependendo das exigências do programa.
    \item Com o Quartus Prime Lite aberto, você está agora pronto para começar a criar e simular seus projetos de sistemas digitais.
\end{enumerate}

Este guia deve ajudar você a instalar e iniciar o Quartus Prime Lite 18.1 em sua máquina Windows com facilidade. Para qualquer suporte adicional ou questões técnicas, referencie a documentação oficial do Quartus ou os fóruns de suporte da Intel.


\subsection{Sobre LaTeX}
Texto sobre LaTeX e como ele é útil.

\section{Inserção de Imagens}
Para inserir uma imagem, você pode usar o seguinte comando:

\begin{figure}[H]
    \centering
    %\includegraphics[width=0.8\textwidth]{nome_da_imagem.extensão}
    \caption{Descrição da imagem.}
    \label{fig:imagem1}
\end{figure}

\section{Conclusão}
Texto de conclusão da sua apostila.

\end{document}
